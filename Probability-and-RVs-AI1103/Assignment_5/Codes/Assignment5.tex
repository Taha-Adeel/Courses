\documentclass[journal,12pt,twocolumn]{IEEEtran}

\usepackage{setspace}
%\usepackage{gensymb}
\singlespacing
\usepackage[cmex10]{amsmath}

\usepackage{amsthm}

\usepackage{mathrsfs}
\usepackage{txfonts}
\usepackage{stfloats}
\usepackage{bm}
\usepackage{cite}
\usepackage{cases}
\usepackage{subfig}
\usepackage{float}
\usepackage{longtable}
\usepackage{multirow}

\usepackage{enumitem}
\usepackage{mathtools}
\usepackage{steinmetz}
\usepackage{tikz}
%\usepackage{circuitikz}
\usepackage{verbatim}
\usepackage{tfrupee}
\usepackage[breaklinks=true]{hyperref}
\usepackage{graphicx}
\usepackage{tkz-euclide}

\usetikzlibrary{calc,math}
\usepackage{listings}
    \usepackage{color}                                            %%
    \usepackage{array}                                            %%
    \usepackage{longtable}                                        %%
    \usepackage{calc}                                             %%
    \usepackage{multirow}                                         %%
    \usepackage{hhline}                                           %%
    \usepackage{ifthen}                                           %%
    \usepackage{lscape}     
\usepackage{multicol}
\usepackage{chngcntr}

\DeclareMathOperator*{\Res}{Res}

\renewcommand\thesection{\arabic{section}}
\renewcommand\thesubsection{\thesection.\arabic{subsection}}
\renewcommand\thesubsubsection{\thesubsection.\arabic{subsubsection}}

\renewcommand\thesectiondis{\arabic{section}}
\renewcommand\thesubsectiondis{\thesectiondis.\arabic{subsection}}
\renewcommand\thesubsubsectiondis{\thesubsectiondis.\arabic{subsubsection}}


\hyphenation{op-tical net-works semi-conduc-tor}
\def\inputGnumericTable{}                                 %%

\lstset{
%language=C,
frame=single, 
breaklines=true,
columns=fullflexible
}
\begin{document}


\newtheorem{theorem}{Theorem}[section]
\newtheorem{problem}{Problem}
\newtheorem{proposition}{Proposition}[section]
\newtheorem{lemma}{Lemma}[section]
\newtheorem{corollary}[theorem]{Corollary}
\newtheorem{example}{Example}[section]
\newtheorem{definition}[problem]{Definition}

\newcommand{\BEQA}{\begin{eqnarray}}
\newcommand{\EEQA}{\end{eqnarray}}
\newcommand{\define}{\stackrel{\triangle}{=}}
\bibliographystyle{IEEEtran}
\raggedbottom
\setlength{\parindent}{0pt}
\providecommand{\mbf}{\mathbf}
\providecommand{\pr}[1]{\ensuremath{\Pr\left(#1\right)}}
\providecommand{\qfunc}[1]{\ensuremath{Q\left(#1\right)}}
\providecommand{\sbrak}[1]{\ensuremath{{}\left[#1\right]}}
\providecommand{\lsbrak}[1]{\ensuremath{{}\left[#1\right.}}
\providecommand{\rsbrak}[1]{\ensuremath{{}\left.#1\right]}}
\providecommand{\brak}[1]{\ensuremath{\left(#1\right)}}
\providecommand{\lbrak}[1]{\ensuremath{\left(#1\right.}}
\providecommand{\rbrak}[1]{\ensuremath{\left.#1\right)}}
\providecommand{\cbrak}[1]{\ensuremath{\left\{#1\right\}}}
\providecommand{\lcbrak}[1]{\ensuremath{\left\{#1\right.}}
\providecommand{\rcbrak}[1]{\ensuremath{\left.#1\right\}}}
\theoremstyle{remark}
\newtheorem{rem}{Remark}
\newcommand{\sgn}{\mathop{\mathrm{sgn}}}
\providecommand{\abs}[1]{(\vert#1)\vert}
\providecommand{\res}[1]{\Res\displaylimits_{#1}} 
\providecommand{\norm}[1]{(\lVert#1)\rVert}
%\providecommand{\norm}[1]{\lVert#1\rVert}
\providecommand{\mtx}[1]{\mathbf{#1}}
\providecommand{\mean}[1]{E([ #1 )]}
\providecommand{\fourier}{\overset{\mathcal{F}}{ \rightleftharpoons}}
%\providecommand{\hilbert}{\overset{\mathcal{H}}{ \rightleftharpoons}}
\providecommand{\system}{\overset{\mathcal{H}}{ \longleftrightarrow}}
	%\newcommand{\solution}[2]{\textbf{Solution:}{#1}}
\newcommand{\solution}{\noindent \textbf{Solution: }}
\newcommand{\cosec}{\,\text{cosec}\,}
\providecommand{\dec}[2]{\ensuremath{\overset{#1}{\underset{#2}{\gtrless}}}}
\newcommand{\myvec}[1]{\ensuremath{\begin{pmatrix}#1\end{pmatrix}}}
\newcommand{\mydet}[1]{\ensuremath{\begin{vmatrix}#1\end{vmatrix}}}
\numberwithin{equation}{subsection}
\makeatletter
\@addtoreset{figure}{problem}
\makeatother
\let\StandardTheFigure\thefigure
\let\vec\mathbf
\renewcommand{\thefigure}{\theproblem}
\def\putbox#1#2#3{\makebox[0in][l]{\makebox[#1][l]{}\raisebox{\baselineskip}[0in][0in]{\raisebox{#2}[0in][0in]{#3}}}}
     \def\rightbox#1{\makebox[0in][r]{#1}}
     \def\centbox#1{\makebox[0in]{#1}}
     \def\topbox#1{\raisebox{-\baselineskip}[0in][0in]{#1}}
     \def\midbox#1{\raisebox{-0.5\baselineskip}[0in][0in]{#1}}
\vspace{3cm}
\title{Assignment 5}
\author{Taha Adeel Mohammed - CS20BTECH11052}
\maketitle
\newpage
\bigskip
\renewcommand{\thefigure}{\theenumi}
\renewcommand{\thetable}{\theenumi}
Download all python codes from 
\begin{lstlisting}
https://github.com/Taha-Adeel/AI1103/tree/main/Assignment_5/Codes
\end{lstlisting}
%
and latex-tikz codes from 
%
\begin{lstlisting}
https://github.com/Taha-Adeel/AI1103/tree/main/Assignment_5
\end{lstlisting}
\section{Problem (UGC/Math (mathA\_Dec 2017) Q.119)}
%%(UGC/MATH (mathA_Dec 2017), Q.119)) \\
Arrival of customers in a shop is a Poisson process with intensity $\lambda =2$. Let $X$ the number of customers entering during the time interval $(1,2)$, and let $Y$ the number of customers entering during the time interval $(5,10)$. Which of the following is true?
\begin{enumerate}[label=(\Alph*)]
    \item$X$ and $Y$ are independent.\\
    \item$X+Y$ is a Poisson with parameter $6$. \\
    \item$X-Y$ is a Poisson with parameter $8$.\\
    \item$\pr{X=0\mid X+Y=12}=\left(\dfrac{5}{6}\right)^{12}$
\end{enumerate}
\section{Solution}
\begin{enumerate}[label=\textbf{(\Alph*)}]
    \item In a Poisson process, the occurrences/frequencies of the event do not depend on past or future occurrences. Therefore $X$ and $Y$ are independent, by definition of Poisson process.
    %Alternatively, using memorylessness property for exponential distributions, we can prove that $X$ and $Y$ are independent.

    Hence option (A) is \textbf{correct.\\}

    \item $X$ and $Y$ are Poisson distributions with parameters $\mu_1 = \lambda \, \tau_1 = 2 \times 1$ and $\mu_2 = \lambda \, \tau_2 = 2 \times 5$ respectively($\tau$ is time-interval). 
    Hence the PMFs (Probability Mass Function) of random variables $X$ and $Y$ are given by:
    \begin{align}
        p_X(x) &= \dfrac{e^{-\mu_1}\mu_1^{x}}{x!}, & \text{for } x=0,1,2,\dots\\
        &= \dfrac{e^{-2}\cdot 2^{x}}{x!}\label{pmf(X)}
    \end{align}
    \begin{align}
        p_Y(y) &= \dfrac{e^{-\mu_2}\mu_2^{y}}{y!}, & \text{for } y=0,1,2,\dots\\
        &= \dfrac{e^{-10}\cdot 10^{y}}{y!}\label{pmf(Y)}
    \end{align}
    As X and Y are independent, we have for $k \geq 0$, the distribution function $p_{X+Y}(k)$ is a convolution of distribution functions $p_X(x)$ and $p_Y(y)$: 
    \begin{align}
       p_{X+Y}(k) &= \pr{X+ Y = k} = \pr{Y = k - X}\\
       &= \sum_{i}\pr{Y = k - i| X=i}\times p_X(i)\label{p_(x+y)}
    \end{align}
    As X and Y are independent: 
    \begin{align}
        \pr{Y\! =\! k - i \mid X=i} = \pr{Y\! = \! k - i} = p_Y(k-i)
    \end{align}
    Simplifying \eqref{p_(x+y)}
    \begin{align}
        p_{X+Y}(k) &= \sum_{i=0}^k p_Y(k-i) \times p_X(i)\\
        &= \sum_{i=0}^k e^{-\mu_2}\frac{\mu_2^{k-i}}{(k-i)!}e^{-\mu_1}\frac{\mu_1^i}{i!}\\
        &= e^{-(\mu_1 + \mu_2)}\frac 1{k!}\sum_{i=0}^k \frac{k!}{i!(k-i)!}\,\mu_1^i\,\mu_2^{k-i}\\
        &= e^{-(\mu_1 + \mu_2)}\frac 1{k!}\sum_{i=0}^k {k\choose i}\, \mu_1^i\,\mu_2^{k-i}\\
        p_{X+Y}(k) &= \frac{e^{-(\mu_1 + \mu_2)} \cdot (\mu_1 + \mu_2)^k}{k!} \\
        \therefore p_{X+Y}(k) &= \frac{e^{-(12)} \cdot (12)^k}{k!}\label{pmf(X+Y)}
    \end{align}
    $\Rightarrow$  $X+ Y$ is a Poisson with parameter $\mu_1 + \mu_2 = 12 \neq 6$. Hence option (B) is \textbf{incorrect.}\\
    \item The distribution function for $X - Y$ no longer remains Poisson, as $X - Y$ will also attain negative values.
    %More precisely, pmf of $X-Y$ is given by \eqref{pmf(X-Y)}, which is a Skellam distribution. 
    %\begin{align}
    %\pr{X-Y=k}=p_{X-Y}(k)=\sum_{i\,=\,-\infty}^\infty p_X(k+i) \times p_Y(i)\label{pmf(X-Y)}
    %\end{align}
    Hence option (C) is \textbf{incorrect.}

    \item \begin{align}
        \pr{X=0\mid X+Y=12} = \dfrac{\pr{X=0, Y=12}}{X+Y=12}
    \end{align}
    As $X$ and $Y$ are independent, and using \eqref{pmf(X)}, \eqref{pmf(Y)}, and \eqref{pmf(X+Y)}, we have:
    \begin{align}
        \pr{X\!=\!0\mid X\!+\!Y\!=\!12} &= \frac{\pr{X\!=\!0} \cdot \pr{Y\!=\!12}}{\pr{X+Y=12}} \\
        &= \frac{\dfrac{e^{-2}\times 2^{0}}{0!} \times \dfrac{e^{-10}\times10^{12}}{12!}}{\dfrac{e^{-12}\times12^{12}}{12!}}\\
        &= \left(\dfrac{5}{6}\right)^{12}
    \end{align}
    Hence option (D) is \textbf{correct.}
\end{enumerate}
\textbf{Ans: (A), (D)}

\end{document}
