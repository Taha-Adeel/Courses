\documentclass[journal,12pt,twocolumn]{IEEEtran}

\usepackage{setspace}
%\usepackage{gensymb}
\singlespacing
\usepackage[cmex10]{amsmath}

\usepackage{amsthm}

\usepackage{mathrsfs}
\usepackage{txfonts}
\usepackage{stfloats}
\usepackage{bm}
\usepackage{cite}
\usepackage{cases}
\usepackage{subfig}
\usepackage{float}
\usepackage{longtable}
\usepackage{multirow}

\usepackage{enumitem}
\usepackage{mathtools}
\usepackage{steinmetz}
\usepackage{tikz}
%\usepackage{circuitikz}
\usepackage{verbatim}
\usepackage{tfrupee}
\usepackage[breaklinks=true]{hyperref}
\usepackage{graphicx}
\usepackage{tkz-euclide}

\usetikzlibrary{calc,math}
\usepackage{listings}
    \usepackage{color}                                            %%
    \usepackage{array}                                            %%
    \usepackage{longtable}                                        %%
    \usepackage{calc}                                             %%
    \usepackage{multirow}                                         %%
    \usepackage{hhline}                                           %%
    \usepackage{ifthen}                                           %%
    \usepackage{lscape}     
\usepackage{multicol}
\usepackage{chngcntr}

\DeclareMathOperator*{\Res}{Res}

\renewcommand\thesection{\arabic{section}}
\renewcommand\thesubsection{\thesection.\arabic{subsection}}
\renewcommand\thesubsubsection{\thesubsection.\arabic{subsubsection}}

\renewcommand\thesectiondis{\arabic{section}}
\renewcommand\thesubsectiondis{\thesectiondis.\arabic{subsection}}
\renewcommand\thesubsubsectiondis{\thesubsectiondis.\arabic{subsubsection}}


\hyphenation{op-tical net-works semi-conduc-tor}
\def\inputGnumericTable{}                                 %%

\lstset{
%language=C,
frame=single, 
breaklines=true,
columns=fullflexible
}
\begin{document}


\newtheorem{theorem}{Theorem}[section]
\newtheorem{problem}{Problem}
\newtheorem{proposition}{Proposition}[section]
\newtheorem{lemma}{Lemma}[section]
\newtheorem{corollary}[theorem]{Corollary}
\newtheorem{example}{Example}[section]
\newtheorem{definition}[problem]{Definition}

\newcommand{\BEQA}{\begin{eqnarray}}
\newcommand{\EEQA}{\end{eqnarray}}
\newcommand{\define}{\stackrel{\triangle}{=}}
\bibliographystyle{IEEEtran}
\raggedbottom
\setlength{\parindent}{0pt}
\providecommand{\mbf}{\mathbf}
\providecommand{\pr}[1]{\ensuremath{\Pr\left(#1\right)}}
\providecommand{\qfunc}[1]{\ensuremath{Q\left(#1\right)}}
\providecommand{\sbrak}[1]{\ensuremath{{}\left[#1\right]}}
\providecommand{\lsbrak}[1]{\ensuremath{{}\left[#1\right.}}
\providecommand{\rsbrak}[1]{\ensuremath{{}\left.#1\right]}}
\providecommand{\brak}[1]{\ensuremath{\left(#1\right)}}
\providecommand{\lbrak}[1]{\ensuremath{\left(#1\right.}}
\providecommand{\rbrak}[1]{\ensuremath{\left.#1\right)}}
\providecommand{\cbrak}[1]{\ensuremath{\left\{#1\right\}}}
\providecommand{\lcbrak}[1]{\ensuremath{\left\{#1\right.}}
\providecommand{\rcbrak}[1]{\ensuremath{\left.#1\right\}}}
\theoremstyle{remark}
\newtheorem{rem}{Remark}
\newcommand{\sgn}{\mathop{\mathrm{sgn}}}
\providecommand{\abs}[1]{(\vert#1)\vert}
\providecommand{\res}[1]{\Res\displaylimits_{#1}} 
\providecommand{\norm}[1]{(\lVert#1)\rVert}
%\providecommand{\norm}[1]{\lVert#1\rVert}
\providecommand{\mtx}[1]{\mathbf{#1}}
\providecommand{\mean}[1]{E([ #1 )]}
\providecommand{\fourier}{\overset{\mathcal{F}}{ \rightleftharpoons}}
%\providecommand{\hilbert}{\overset{\mathcal{H}}{ \rightleftharpoons}}
\providecommand{\system}{\overset{\mathcal{H}}{ \longleftrightarrow}}
	%\newcommand{\solution}[2]{\textbf{Solution:}{#1}}
\newcommand{\solution}{\noindent \textbf{Solution: }}
\newcommand{\cosec}{\,\text{cosec}\,}
\providecommand{\dec}[2]{\ensuremath{\overset{#1}{\underset{#2}{\gtrless}}}}
\newcommand{\myvec}[1]{\ensuremath{\begin{pmatrix}#1\end{pmatrix}}}
\newcommand{\mydet}[1]{\ensuremath{\begin{vmatrix}#1\end{vmatrix}}}
\numberwithin{equation}{subsection}
\makeatletter
\@addtoreset{figure}{problem}
\makeatother
\let\StandardTheFigure\thefigure
\let\vec\mathbf
\renewcommand{\thefigure}{\theproblem}
\def\putbox#1#2#3{\makebox[0in][l]{\makebox[#1][l]{}\raisebox{\baselineskip}[0in][0in]{\raisebox{#2}[0in][0in]{#3}}}}
     \def\rightbox#1{\makebox[0in][r]{#1}}
     \def\centbox#1{\makebox[0in]{#1}}
     \def\topbox#1{\raisebox{-\baselineskip}[0in][0in]{#1}}
     \def\midbox#1{\raisebox{-0.5\baselineskip}[0in][0in]{#1}}
\vspace{3cm}
\title{Assignment 3}
\author{Taha Adeel Mohammed - CS20BTECH11052}
\maketitle
\newpage
\bigskip
\renewcommand{\thefigure}{\theenumi}
\renewcommand{\thetable}{\theenumi}
Download all python codes from 
\begin{lstlisting}
https://github.com/Taha-Adeel/AI1103/tree/main/Assignment%201/codes
\end{lstlisting}
%
and latex-tikz codes from 
%
\begin{lstlisting}
https://github.com/Taha-Adeel/AI1103/tree/main/Assignment%201
\end{lstlisting}
\section{Problem (GATE 2008 (CS), Q.27)}
%%(GATE 2008 (CS), Q.27) \\
Aishwarya studies either computer science or mathematics everyday. If she studies computer science on a day, then the probability she studies mathematics the next day is 0.6. If she studies mathematics on a day, then the probability she studies computer science the next day is 0.4.
Given that Aishwarya studies computer science on Monday, what is the probablity she studies computer science on Wednesday?
\begin{enumerate}[label=(\Alph*)]
\begin{multicols}{2}
%\setlength\itemsep{2em}
\item 0.24
\item 0.36
\item 0.4
\item 0.6
\end{multicols}
\end{enumerate}
\section{Solution}
Let the random variable $X_i \in \{ 0,1 \}$ , $i=0,1,2, \cdots$ represent her studying CS(Computer Science) or mathematics respectively on the $i$th day.
\begin{table}[h]
    \centering
    \begin{tabular}[width=\columnwidth]{|c|c|c|c|}
         \hline
        \textbf{Subject}&\textbf{X$_i$}&\textbf{$\pr{X_i | X_{i-1}=0}$} &\textbf{$\pr{X_i | X_{i-1}=1}$} \\
        \hline    
         CS&0&x (Ref \eqref{x})&0.4\\
         \hline
         Maths&1&0.6&y (Ref \eqref{y})\\
         \hline
    \end{tabular}
\end{table}
\par As $X_i=0 \text{ and } X_i=1$ are mutually exclusive, we can easily calculate $x$ and $y$.
\begin{align}
    x=\pr{X_{i} = 0 \,|\, X_{i-1}=0} &= 1-\pr{X_{i} = 1 \,|\, X_{i-1}=0}\nonumber
    \\&= 0.4 \label{x}\\
    y=\pr{X_{i} = 1 \,|\, X_{i-1}=1} &= 1-\pr{X_{i} = 0 \,|\, X_{i-1}=1}\nonumber
    \\&= 0.6 \label{y}
\end{align} 
\begin{align}
    \therefore \pr{X_i=0}&=\pr{X_{i} = 0 \,|\, X_{i-1}=0}\times \pr{ X_{i-1}=0}\nonumber\\
    &+ \pr{X_{i} = 0 \,|\, X_{i-1}=1}\times \pr{ X_{i-1}=1}  \nonumber\\
&=0.4\times\left( \pr{ X_{i-1}=0}+\pr{ X_{i-1}=1} \right) \nonumber\\
&=0.4 \times 1 = 0.4
\end{align}
Therefore, the probability that Aishwarya studies computer science on any day is $0.4$. So, the probablity she studies computer science on Wednesday is also 0.4\\
 (\textbf{Ans: Option (C)})
\end{document}
