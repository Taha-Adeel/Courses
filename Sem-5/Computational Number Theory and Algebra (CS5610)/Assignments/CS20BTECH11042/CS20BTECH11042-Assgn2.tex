%%%%%%%%%%%%%%%%%%%%%%%%%%%%% Define Article %%%%%%%%%%%%%%%%%%%%%%%%%%%%%%%%%%
\documentclass{article}
%%%%%%%%%%%%%%%%%%%%%%%%%%%%%%%%%%%%%%%%%%%%%%%%%%%%%%%%%%%%%%%%%%%%%%%%%%%%%%%

%%%%%%%%%%%%%%%%%%%%%%%%%%%%% Using Packages %%%%%%%%%%%%%%%%%%%%%%%%%%%%%%%%%%
\usepackage{geometry}
\usepackage{graphicx}
\usepackage{amssymb}
\usepackage{amsmath}
\usepackage{amsthm}
\usepackage{empheq}
\usepackage{mdframed}
\usepackage{booktabs}
\usepackage{lipsum}
\usepackage{graphicx}
\usepackage{color}
\usepackage{psfrag}
\usepackage{pgfplots}
\usepackage{bm}
%%%%%%%%%%%%%%%%%%%%%%%%%%%%%%%%%%%%%%%%%%%%%%%%%%%%%%%%%%%%%%%%%%%%%%%%%%%%%%%

% Other Settings

%%%%%%%%%%%%%%%%%%%%%%%%%% Page Setting %%%%%%%%%%%%%%%%%%%%%%%%%%%%%%%%%%%%%%%
\geometry{a4paper}

%%%%%%%%%%%%%%%%%%%%%%%%%% Define some useful colors %%%%%%%%%%%%%%%%%%%%%%%%%%
\definecolor{ocre}{RGB}{243,102,25}
\definecolor{mygray}{RGB}{243,243,244}
\definecolor{deepGreen}{RGB}{26,111,0}
\definecolor{shallowGreen}{RGB}{235,255,255}
\definecolor{deepBlue}{RGB}{61,124,222}
\definecolor{shallowBlue}{RGB}{235,249,255}
%%%%%%%%%%%%%%%%%%%%%%%%%%%%%%%%%%%%%%%%%%%%%%%%%%%%%%%%%%%%%%%%%%%%%%%%%%%%%%%

%%%%%%%%%%%%%%%%%%%%%%%%%% Define an orangebox command %%%%%%%%%%%%%%%%%%%%%%%%
\newcommand\orangebox[1]{\fcolorbox{ocre}{mygray}{\hspace{1em}#1\hspace{1em}}}
%%%%%%%%%%%%%%%%%%%%%%%%%%%%%%%%%%%%%%%%%%%%%%%%%%%%%%%%%%%%%%%%%%%%%%%%%%%%%%%

%%%%%%%%%%%%%%%%%%%%%%%%%%%% English Environments %%%%%%%%%%%%%%%%%%%%%%%%%%%%%
\newtheoremstyle{mytheoremstyle}{3pt}{3pt}{\normalfont}{0cm}{\rmfamily\bfseries}{}{1em}{{\color{black}\thmname{#1}~\thmnumber{#2}}\thmnote{\,--\,#3}}
\newtheoremstyle{myproblemstyle}{3pt}{3pt}{\normalfont}{0cm}{\rmfamily\bfseries}{}{1em}{{\color{black}\thmname{#1}~\thmnumber{#2}}\thmnote{\,--\,#3}}
\theoremstyle{mytheoremstyle}
\newmdtheoremenv[linewidth=1pt,backgroundcolor=shallowGreen,linecolor=deepGreen,leftmargin=0pt,innerleftmargin=20pt,innerrightmargin=20pt,]{theorem}{Theorem}[section]
\theoremstyle{mytheoremstyle}
\newmdtheoremenv[linewidth=1pt,backgroundcolor=shallowBlue,linecolor=deepBlue,leftmargin=0pt,innerleftmargin=20pt,innerrightmargin=20pt,]{definition}{Definition}[section]
\theoremstyle{myproblemstyle}
\newmdtheoremenv[linecolor=black,leftmargin=0pt,innerleftmargin=10pt,innerrightmargin=10pt,]{problem}{Problem}[section]
%%%%%%%%%%%%%%%%%%%%%%%%%%%%%%%%%%%%%%%%%%%%%%%%%%%%%%%%%%%%%%%%%%%%%%%%%%%%%%%

%%%%%%%%%%%%%%%%%%%%%%%%%%%%%%% Plotting Settings %%%%%%%%%%%%%%%%%%%%%%%%%%%%%
\usepgfplotslibrary{colorbrewer}
\pgfplotsset{width=8cm,compat=1.9}
%%%%%%%%%%%%%%%%%%%%%%%%%%%%%%%%%%%%%%%%%%%%%%%%%%%%%%%%%%%%%%%%%%%%%%%%%%%%%%%

%%%%%%%%%%%%%%%%%%%%%%%%%%%%%%% Title & Author %%%%%%%%%%%%%%%%%%%%%%%%%%%%%%%%
\title{Computational Number Theory}
\author{HW 2}
\date{CS20BTECH11042}
%%%%%%%%%%%%%%%%%%%%%%%%%%%%%%%%%%%%%%%%%%%%%%%%%%%%%%%%%%%%%%%%%%%%%%%%%%%%%%%

\begin{document}
    \maketitle
    
\section{Question 1}
\begin{itemize}
    \item Using Chinese Remainder Theorem, we can say that solutions of $x^2 -1$ in $\mathbb{Z}_{17}$ and $x^2 -1$ in $\mathbb{Z}_{19}$ are the solutions we require.
    \item  $\mathbb{Z}_{17}$
    \begin{align*}
        (x-1)(x+1) = 0 \mod 17 \\
        \Rightarrow x = 1, 16 \mod 17
    \end{align*}
    \item  $\mathbb{Z}_{19}$
    \begin{align*}
        (x-1)(x+1) = 0 \mod 19 \\
        \Rightarrow x = 1, 18 \mod 19
    \end{align*}
    \item  $\mathbb{Z}_{17 \times 19}$
    \begin{itemize}
        \item $x = 1 \mod 17$ and $x = 1 \mod 19 \Rightarrow x = 1 \mod {17 \times 19}$
        \item $x = 16 \mod 17$ and $x = 1 \mod 19 \Rightarrow x = 305 \mod {17 \times 19}$
        \item $x = 1 \mod 17$ and $x = 18 \mod 19 \Rightarrow x = 18 \mod {17 \times 19}$
        \item $x = 16 \mod 17$ and $x = 18 \mod 19 \Rightarrow x = 322 \mod {17 \times 19}$
    \end{itemize}
    \item Therefore, the roots of $x^2 -1$ in $\mathbb{Z}_{17 \times 19}$ are $\{1, 18, 305, 322\}$
\end{itemize}


\section{Question 2}
    \begin{itemize}
        \item We observe that 41 is a prime number and $7 \nmid 41$
        \item Using Euclids lemma, we find k=23 satisfies the equation, $7k = 1 \mod 40$
        \item Now, raising both sides of $x^7 = 2 \mod 41$ to the power of 23, we get,
        \begin{align*}
            x^{7 \times 23} &= 2^{23} \mod 41 \\
            x^{161} &= 2^{23} \mod 41 \\
            x &= 2^{23} \mod 41 \ (\because \text{Fermat's Little Theorem})\\
            x &= 8 \mod 41 
        \end{align*}
    \end{itemize}

\section{Question 3}
\begin{itemize}
    \item Given, p is an odd prime number and $d | (p-1)$
    \item Let $S : \{a \in \mathbb{Z}_p : a^d = 1\}$
    \item Let $T : \{a^{(p-1)/d} : a \in \mathbb{Z}_p\}$. Here, since p is prime, $\mathbb{Z}_p = \mathbb{Z}^{*}_p$
    \item Now, for every element $x \in T$, we can see that $x^d = {(a^{(p-1)/d})}^d = a^{p-1} = 1$. Therefore, $\forall x \in T, x \in S \Rightarrow T \subset S$
    \item Consider the equation $x^{d} - 1 = 0$, which has $p-1$ roots in $\mathbb{Z}_p$, denoted by the set $S$.
    \item We know that this equation has d distinct roots in $\mathbb{Z}_p$ and we see that 
\end{itemize}
\section{Question 4}
\subsection{Part a}
\begin{itemize}
    \item Given \( dk \equiv 0 \mod n \), it implies \( n \) divides \( dk \). Therefore, \( k \) must be a multiple of \( \frac{n}{\text{gcd}(d, n)} \).
    \item Hence, let \( f(d) \) represents the count of multiples of \( \frac{n}{\text{gcd}(d, n)} \) within the range \( 0 \leq k \leq n-1 \).
    \item The count of multiples of \( m \) within \( 0 \leq k \leq n-1 \) is given by \( \left\lfloor \frac{n}{m} \right\rfloor \).
    \item \( \implies f(d) = \left\lfloor \frac{n}{\frac{n}{\text{gcd}(d, n)}} \right\rfloor = \text{gcd}(d, n) \).
    \item Therefore, \( |\{ 0 \leq k \leq n-1 : dk \equiv 0 \mod n \}| = \text{gcd}(d, n) \).
\end{itemize}

\subsection{Part b}
\begin{itemize}
    \item Given that $x^d = 1 \mod p$
    \item We raise both sides by k to get: $x^{dk} = 1 \mod p$
    \item $\implies dk = \gcd(d, p - 1) \mod p - 1\$ 
    \item From part a, we know that the number of solutions to above equation is $\gcd(d, p - 1)$
    \item That is if we find an arbitrary solution as a primitive root, then the rest of the solutions are powers of the primitive root.
    \item Therefore, the number of roots of $x^d - 1$ in Zp is $\gcd(d, p - 1)$
\end{itemize}

\section{Question 5}
\begin{itemize}
    \item We need to find the roots of the equation $x^2 - 4$ in $\mathbb{Z}_{343}$
    \item $\mathbb{Z}_{7}$
    \begin{align*}
        (x-2)(x+2) = 0 \mod 7 \\
        \Rightarrow x = 2, 5 \mod 7
    \end{align*}
    \item  $\mathbb{Z}_{49}$, we use \textbf{Hensel Lifting},
    \begin{itemize}
        \item Let x be of the form $7y+b$ where $b \in \{2,5\}$
        \item Now, when b = 2,
        \begin{align*}
            x^2 &= 4 \mod 49 \\
            \Rightarrow {(7y+2)}^2 &= 4\mod 49 \\
            \Rightarrow 28y + 4 &= 4 \mod 49 \\
            \Rightarrow 28y &= 0 \mod 49 \\
            \Rightarrow y &= 7k, k \in \mathbb{Z} \\
            \Rightarrow x &= 2
        \end{align*}
        \item Now, when b = 5,
        \begin{align*}
            x^2 &= 4 \mod 49 \\
            \Rightarrow {(7y+5)}^2 &= 4\mod 49 \\
            \Rightarrow 70y + 25 &= 4 \mod 49 \\
            \Rightarrow 70y &= -21 \mod 49 \\
            \Rightarrow 10y &= -3 \mod 7 \\
            \Rightarrow y &= 7k +6 \\
            \Rightarrow x &= 47
        \end{align*}
    \end{itemize}
    \item $\mathbb{Z}_{343}$, we use \textbf{Hensel Lifting} again
    \begin{itemize}
        \item Let x be of the form $49y+b$ where $b \in \{2,47\}$
        \item Now, when b = 2,
        \begin{align*}
            x^2 &= 4 \mod 343 \\
            \Rightarrow {(49y+2)}^2 &= 4\mod 343 \\
            \Rightarrow 98y + 4 &= 4 \mod 343 \\
            \Rightarrow 98y &= 0 \mod 343 \\
            \Rightarrow y &= 7k, k \in \mathbb{Z} \\
            \Rightarrow x &= 2
        \end{align*}
        \item Now, when b = -2, we can see that $x = 341 \mod 343$
    \end{itemize}
\end{itemize}

Therefore, the roots of $x^2 - 4$ in $\mathbb{Z}_{343}$ are $\{2, 341\}$

\end{document}