\documentclass[12pt]{article}
\usepackage[reqno]{amsmath}
\usepackage{amsfonts}
\usepackage{mathtools}
\usepackage{amssymb}
\usepackage{amsthm}
%\usepackage{multicol}
\usepackage[a4paper, margin=1in]{geometry}
\usepackage{indentfirst}
\usepackage{bm}
\usepackage{wrapfig}
\usepackage{enumitem}
\usepackage{mwe}
\usepackage{hyperref}
\hypersetup{
    colorlinks=true,
    linkcolor=blue,      
    urlcolor=cyan,
}
\renewcommand{\thesection}{\Roman{section}}
\setlength{\parindent}{4em}


\begin{document}
\begin{center}
\par\noindent\rule{\textwidth}{0.6pt}\\[0.3cm]
\textbf{\LARGE{Assignment 4}}\\[0.3cm]
\Large{BT1010 : Introduction to Life Sciences}\\[0.1cm]
\large{April 7, 2021}\\[0cm]
\par\noindent\rule{\textwidth}{0.6pt}
\end{center}
%\begin{multicols}{2}
\noindent
\hspace{0.4cm}Name : Taha Adeel Mohammed
\par \noindent
\hspace{0.4cm}Roll No. : CS20BTECH11052
\section*{Question}
\noindent Why only D sugars and L amino acids are found in all the organism on earth?

\section*{Answer}
\noindent One possible theory is that in the prebiotic world, amino acids brought along with meteors to Earth had a small extent of L-amino acids, whose ratio in solutions could be greatly amplified. In these solutions, life-forms that used the dominant L-amino acids would prevail due to natural selection. L-amino acids also catalyzed the formation of a greater percent of D-sugars compared to L-sugars. Hence the earliest lifeforms contained only L-amino acids and D-sugars, which they passed on to all organisms on Earth through the process of evolution. 
\end{document}