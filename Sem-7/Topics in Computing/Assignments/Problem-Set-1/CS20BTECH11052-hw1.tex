\documentclass{article}
\usepackage[utf8]{inputenc}
\usepackage{geometry}
\usepackage{fancyhdr}
\usepackage{titling}
\usepackage{datetime}
\usepackage{graphicx}
\usepackage{lipsum}
\usepackage{titlesec}
\usepackage{amsmath}
\usepackage{amssymb}
\usepackage{amsthm}
\usepackage{blkarray}

\usepackage{tikz}
\usetikzlibrary{automata, positioning}

% Set your name and assignment details here
\renewcommand{\author}{Taha Adeel Mohammed}
\newcommand{\rollnumber}{CS20BTECH11052}
\newcommand{\course}{CS5160: Topics in Computing}
\newcommand{\assignment}{Problem Set 1}

\renewcommand{\S}{\mathcal{S}}
\newcommand{\A}{\mathcal{A}}
\renewcommand{\P}{\mathcal{P}}
\newcommand{\R}{\mathcal{R}}
\newcommand{\Rhat}{\hat{\mathcal{R}}}

% \renewcommand{\thesection}{Problem \arabic{section}:\!\!\!\!}
\renewcommand{\thesubsection}{(\arabic{subsection})\!\!\!}
% \titleformat{\subsection}{\normalfont}{\thesubsection}{0.5em}{}

% Page setup
\geometry{a4paper, margin=1in}
\lfoot{\myname}
\rfoot{AI3000/CS5500}
\cfoot{\assignment}
\rfoot{\thepage}

% Title
\renewcommand{\maketitle}{
	\begin{center}
		\line(1,0){450} \\
		\vspace*{1ex}
        \Large{\textbf{\course}} \\
        \Large{\textbf{\assignment}} \\
    \end{center}
	\large{\author}
	\begin{flushright}
		\vspace*{-5ex}
		\rollnumber \\
	\end{flushright}
	\begin{center}
		\vspace*{-1ex}
		\line(1,0){450}
	\end{center}
}

\begin{document}

\maketitle

\textit{(Crediting the course)}

\subsection{\boldmath{Let $f : \{0, 1\}^n \rightarrow \{0, 1\}$ be a Boolean function, and $k$ be the largest natural number such that $|f^{-1}(1)|$ is divisible by $2^k$. Show that $D(f) \geq n - k$, where $D(f)$ is the decision tree complexity of $f$.}}

\textbf{Proof by contradiction:}

% Frame the solution properly
Assume that $D(f) < n - k$, i.e. $D(f) \leq n - k - 1$. Then, there exists a decision tree $T$ of depth atmost $n - k - 1$ that computes $f$.

\noindent
Let $P$ be a path from the root to a leaf node with value $1$ in $T$. Then the path $P$ would query atmost $n - k - 1$ bits. This implies atleast $k + 1$ bits in the input can be varied while still following the same path. Hence, there exists $2^{k+1+t}$ inputs, where $t \geq 0$, that follow the same path $P$. 

\noindent
To get $|f^{-1}(1)|$, we sum the number of inputs for all paths $P$ that lead to a leaf node with value $1$. Hence, 
\begin{align*}
	|f^{-1}(1)| &= \sum_{P} 2^{k+1+t} \\
	&= 2^{k+1} \sum_{P} 2^t \\
 \implies |f^{-1}(1)| &= 2^{k+1} \cdot q, \text{ where } q \in \mathbb{N}
\end{align*}

\noindent
This implies that $|f^{-1}(1)|$ is divisible by $2^{k+1}$, which contradicts the fact that $k$ is the largest natural number such that $|f^{-1}(1)|$ is divisible by $2^k$. Hence, our assumption that $D(f) < n - k$ is false, and hence $$\boxed{D(f) \geq n - k}$$.
\vspace*{-8mm}
\begin{flushright}
	\qedsymbol{}
\end{flushright}

\subsection{For every $\mathbf{k}$, define $\mathbf{f_k}$ to be the following function taking inputs of length $\mathbf{n = 2^k}$:
$$ f_k(x_1, \ldots, x_n) = \begin{cases}
	f_{k-1}(x_1, \ldots, x_{2^{k-1}}) \land f_{k-1}(x_{2^{k-1}+1}, \ldots, x_n) & \text{if $k > 0$ is even} \\
	f_{k-1}(x_1, \ldots, x_{2^{k-1}}) \lor f_{k-1}(x_{2^{k-1}+1}, \ldots, x_n) & \text{if $k$ is odd} \\
	x_i & \text{if $k = 0$}
\end{cases} $$
\boldmath{Show that $D(f_k) = 2^k$.}}

\textbf{Proof by induction:}

\textit{\textbf{Induction Hypothesis:}} For $k \geq 0$, $\exists$ distinct inputs $x, y$ s.t for $2^k-1$ bits, $x_i = y_i$, and $f_k(x) = 0$, $f_k(y) = 1$ and $D(f, x) = D(f, y) = 2^k$. This would also imply that $D(f_k) \geq 2^k$, and since $D(f_k) \leq n = 2^k$, we have $D(f_k) = 2^k$. 

\,

\textit{\textbf{Base case:}} For $k = 0$, we have $x = 0$ and $y = 1$ s.t $f_0(x) = 0$, $f_0(y) = 1$, and $D(f_0, x) = D(f_0, y) = 1$. Hence, the base case holds.

\,

\textit{\textbf{Induction step:}} Assume that the induction hypothesis holds for $k = m$. i.e. there exists $x_m$ and $y_m$ s.t for $2^m-1$ bits, $x_{mi} = y_{mi}$, and $f_m(x_m) = 0$, $f_m(y_m) = 1$ and $D(f_m, x_m) = D(f_m, y_m) = 2^m$.
We need to show that it holds for $k = m + 1$.

\,

\noindent
\underline{Case 1:} $m + 1$ is even

This implies that $f_m(x_1, \ldots, x_n) = f_m(x_1, \ldots, x_{n/2}) \land f_m(x_{n/2+1}, \ldots, x_n)$. Here we would have 
\begin{align*}	
	x_{m+1} &= y_m + x_m & (f_{m+1}(x_{m+1}) = f_{m}(x_m) \land f_{m}(y_m) = 1 \land 0 = 0) \\
	y_{m+1} &= y_m + y_m & (f_{m+1}(y_{m+1}) = f_{m}(y_m) \land f_{m}(y_m) = 1 \land 1 = 1)
\end{align*}
Clearly $x_{m+1} \neq y_{m+1}$, and for $2^{m+1}-1$ bits, $x_{(m+1)i} = y_{(m+1)i}$. Also, we have $f_{m+1}(x_{m+1}) = 0$, $f_{m+1}(y_{m+1}) = 1$, and $D(f_{m+1}, x_{m+1}) = D(f_{m+1}, y_{m+1}) = 2^{m+1}$. Hence, the induction step holds for this case.

\,

\noindent
\underline{Case 2:} $m + 1$ is odd

This implies that $f_m(x_1, \ldots, x_n) = f_m(x_1, \ldots, x_{n/2}) \lor f_m(x_{n/2+1}, \ldots, x_n)$. Here we would have
\begin{align*}
	x_{m+1} &= x_m + x_m & (f_{m+1}(x_{m+1}) = f_{m}(x_m) \lor f_{m}(y_m) = 0 \lor 0 = 0) \\
	y_{m+1} &= x_m + y_m & (f_{m+1}(y_{m+1}) = f_{m}(x_m) \lor f_{m}(y_m) = 0 \lor 1 = 1)
\end{align*}
Clearly $x_{m+1} \neq y_{m+1}$, and for $2^{m+1}-1$ bits, $x_{(m+1)i} = y_{(m+1)i}$. Also, we have $f_{m+1}(x_{m+1}) = 0$, $f_{m+1}(y_{m+1}) = 1$, and $D(f_{m+1}, x_{m+1}) = D(f_{m+1}, y_{m+1}) = 2^{m+1}$. Hence, the induction step also holds for this case.

\,

Hence the induction step holds, and hence the induction hypothesis holds for all $k \geq 0$. Therefore we have
$$\boxed{D(f_k) = 2^k}$$
\vspace*{-8mm}\begin{flushright}\qedsymbol{}\end{flushright}


\subsection{\boldmath{Show that for a non-constant symmetric function $f : \{0, 1\}^n \rightarrow \{0, 1\}$, 
$$ s(f) \geq \Big\lceil \frac{n+1}{2} \Big\rceil, $$
where $s(f)$ is the sensitivity of $f$. Also, give an example where this bound is tight. Recall, we say a function is \textit{symmetric} if its value depends only on the number of 1s in the input. That is, it depends on the hamming weight of the input.}}

Since the function is symmetric, we have that $f(x)$ only depends on $|x|$, the hamming weight of $x$. Also, since the function is non-constant, there exist values $k \in [1, n]$ such that for inputs $x, y$, with $|x| = k-1$ and $|y| = k $, $f(x) \neq f(y)$. Here, we would have
\begin{align}
	s(f, x) &\geq n - (k - 1)  &\text{(converting $x$ to $y$ by flipping the 0s)} \\
	s(f, y) &\geq k &\text{(converting $y$ to $x$ by flipping the 1s)}
\end{align}
Therefore we have,
\begin{align*}
	 &\,s(f) \geq \max \{ s(f, x), s(f, y) \} \\
	 &\quad\quad\geq \max \{ n - (k - 1), k \}, \text{ for some $k \in [1, n]$} \\
	 \implies &\boxed{s(f) \geq \Big\lceil \frac{n+1}{2} \Big\rceil}
\end{align*}
\vspace*{-6mm}\begin{flushright}\qedsymbol{}\end{flushright}

\noindent
\textbf{Example:} Consider the function $f : \{0, 1\}^n \rightarrow \{0, 1\}$ defined as
\begin{align}
	f(x) &= \begin{cases}
		1 & \text{if } |x| \geq \Big\lceil \frac{n+1}{2} \Big\rceil = k \\
		0 & \text{otherwise}
	\end{cases}
\end{align}
Then for $x$ s.t $|x| < k - 1$ or $|x| > k$, we would have 
$$s(f, x) = 0.$$ 
For $x$ s.t $|x| = k - 1$, we would have 
\begin{align*}
	s(f, x) &= n - (k - 1) \\
	&= n - \Big\lceil \frac{n+1}{2} \Big\rceil + 1 \\
	&= \Big\lceil \frac{n+1}{2} \Big\rceil
\end{align*}
Similarly for $x$ s.t $|x| = k$, we would have
$$ s(f, x) = k = \Big\lceil \frac{n+1}{2} \Big\rceil $$
Hence, we have that $s(f) = \Big\lceil \frac{n+1}{2} \Big\rceil$. Therefore, the bound is tight.
\vspace*{-2mm}\begin{flushright}\qedsymbol{}\end{flushright}

\,
\subsection{\boldmath{Show that $s(f) = bs(f) = C(f)$ for every monotone Boolean function $f$, where $bs(f)$ and $C(f)$ respectively denote the block sensitivity and certificate complexity of $f$. Recall we say a function is monotone if for any $x, y$ such that $x \leq y$, that is, $x_i \leq y_i$ for all $i \in [n]$, $f(x) \leq f(y)$.}}

In class we have seen that for all $x \in \{0, 1\}^n$,
\begin{align}
	s(f, x) \leq bs(f, x) &\leq C(f, x) \leq D(f, x)
\end{align}
This implies that
$$ s(f) \leq bs(f) \leq C(f) $$

\noindent
Hence, it is sufficient to show that $s(f) = C(f)$.


\,
\subsection{\boldmath{Prove that when $f$ is a monotone, $s(f) \leq \text{deg}(f)$.}}


\,
\subsection{\boldmath{Define \textit{average sensitivity}, $as(f)$, of a Boolean function $f$ to be the expected sensitivity of $f$ on a random input: $as(f) = \frac{1}{2^n} \sum_{x \in \{0, 1\}^n} s(f, x)$. Let $T$ be a decision tree and let $\ell_x$ be the length of the unique path in $T$ consistent with $x$. Define \textit{average depth} of $T$ to be $\frac{1}{2^n} \sum_{x \in \{0, 1\}^n} \ell_x$. Show that the average depth of any decision tree for $f$ is at least $as(f)$.}}
In class, we have seen the following results for all $x \in \{0, 1\}^n$:
\begin{align}
	C(f, x) &\leq D(f, x), \\
	s(f, x) \leq bs(f, x) &\leq C(f, x) \leq D(f),
\end{align}
where $C$ is the certificate complexity, $D$ is the decision tree complexity, $s$ is the sensitivity, and $bs$ is the block sensitivity.

\,\\
\noindent
Combining these results, we have
\begin{align}
	s(f, x) &\leq D(f, x) = l_x, \ \forall \ x \in \{0, 1\}^n
\end{align}
Hence, we have
\begin{align*}
	\sum_{x \in \{0, 1\}^n} s(f, x) &\leq \sum_{x \in \{0, 1\}^n} l_x \\
	\implies \frac{1}{2^n} \cdot as(f) &\leq \frac{1}{2^n} \sum_{x \in \{0, 1\}^n} l_x \\
	\implies\quad\ \  as(f) &\leq average\ depth\ of\ T
\end{align*}
Therefore the average depth of any decision tree for $f$ is at least $as(f)$.
\vspace*{-2mm}\begin{flushright}\qedsymbol{}\end{flushright}

\end{document}